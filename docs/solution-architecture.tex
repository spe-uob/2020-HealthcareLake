\documentclass[10pt]{article}
\usepackage[hmarginratio=1:1,top=32mm,columnsep=20pt]{geometry}                       			


\title{Healthcare Data Lake}

\author{
	\textsc{Kendal, Joseph}\\
	\normalsize University of Bristol\\
	\texttt{jk17246@bristol.ac.uk}
	
	\and
	
	\textsc{Sherred, Jago}\\
	\normalsize University of Bristol\\
	\texttt{j.sherred.2019@bristol.ac.uk}
	
	\and
	
	\textsc{Benson, Luke}\\
	\normalsize University of Bristol\\
	\texttt{wr19606@bristol.ac.uk}
	
	\and
	
	\textsc{Liu, Anna}\\
	\normalsize University of Bristol\\
	\texttt{gf19916@bristol.ac.uk}
	
	\and
	
	\textsc{Cismaru, Armand}\\
	\normalsize University of Bristol\\
	\texttt{fz19792@bristol.ac.uk}
}


\begin{document}

\maketitle    


\begin{abstract}

Digital healthcare provided by the NHS in England typically operates in silos. GPs have electronic systems to manage patient care which are distinct from hospital systems which are distinct from the ambulance service, 111, mental health services etc. Each data owner has a wealth of data that, if combined, would generate a more valuable resource than it does in isolation. While there are solutions to integrate this data for direct care purposes, there is no centralised solution to use this data to inform future care or service provisioning. This project is designed to explore the benefits of cloud technologies to produce a prototype secure, scalable health data storage platform that can underpin local healthcare analytics.

\end{abstract}

\section{Overview}
%TODO%

\subsection{Client}
%TODO%
\subsection{Domain}
%TODO%
\subsection{Project}
%TODO%
\subsection{Vision}

\newpage


\section{Requirements}
\subsection{Stakeholders}
\subsection{User stories}


\section{Personal Data, Privacy, Security and Ethics Management}
\subsection{GDPR}
\subsection{Security}
\subsection{Ethics}


\section{Architecture}


\section{Development Testing}


\section{Release Testing}


\section{OO Design \& UML}


\end{document}
